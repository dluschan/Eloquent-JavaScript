\documentclass[13pt,oneside]{scrbook}
\usepackage{natbib}
\usepackage{xltxtra}
\setmainfont{XCharter}
\setsansfont{Liberation Sans}
\newfontfamily{\cyrillicfonttt}{Inconsolata LGC}
\defaultfontfeatures{Mapping=tex-text}

\usepackage{polyglossia}
\setdefaultlanguage{russian}

\usepackage{hyperref}
\usepackage{bookmark}
\usepackage{listings}
\usepackage{graphicx}
\usepackage{makeidx}

% epigraph is used to style chapter quotes
\usepackage{epigraph}
\setlength{\epigraphwidth}{.8\textwidth}
\setlength{\epigraphrule}{0pt}

\lstset{basicstyle=\ttfamily,xleftmargin=0.8em,breaklines=true,lineskip=-0.2em,aboveskip=0.8em,belowskip=0.8em,keepspaces=true}
\date{}

\makeatletter
\setcounter{secnumdepth}{0}
\setcounter{tocdepth}{1}
\setmonofont[Scale=0.8]{Inconsolata LGC}
\pagestyle{plain}
\makeatother

\usepackage{newunicodechar}
\newunicodechar{→}{{\tiny$\rightarrow$\normalsize}}
\newunicodechar{←}{$\leftarrow$}
\newunicodechar{“}{``}
\newunicodechar{”}{''}
\newunicodechar{’}{'}
\newunicodechar{π}{$\pi$}
\newunicodechar{½}{$\frac{1}{2}$}
\newunicodechar{⅓}{$\frac{1}{3}$}
\newunicodechar{¼}{$\frac{1}{4}$}
\newunicodechar{…}{...}
\newunicodechar{×}{$\times$}
\newunicodechar{β}{\ss}
\newunicodechar{ϕ}{$\varphi$}
\newunicodechar{≈}{$\approx$}
\newunicodechar{—}{---}

\graphicspath{ {../} }
\makeindex

\begin{document}

\author{Марейн Хавербек}

\title{Выразительный JavaScript}

\subtitle{Современное введение в программирование}

\maketitle

\frontmatter

  \noindent Copyright \textcopyright{} 2014 Марейн Хавербек

  \vskip 1em

  \noindent Эта работа распространяется под некоммерческой лицензией Creative Commons с указанием авторства (\url{http://creativecommons.org/licenses/by-nc/3.0/}). Весь код в этой книге также может считаться лицензированным по лицензии MIT (\url{http://opensource.org/licenses/MIT}).

  Иллюстрации предоставлены разными художниками: обложка от Васифа Хайдера (Wasif Hyder), компьютер (введение) и одноколёсные велосипеды (глава 21) от Макса Сяньту (Max Xiantu), море битов (глава 1) и белка-оборотень (глава 4) от Маргариты Мартинес (Margarita Martínez) и Хосе Менора (José Menor), осьминоги (главы 2 и 4) от Джима Тирни (Jim Tierney), объект с выключателем (глава 6) от Дайла МакГрегора (Dyle MacGregor). Диаграммы регулярных выражений в главе 9 созданы с помощью \href{http://regexper.com}{regexper.com} Джеффом Аваллоном (Jeff Avallone). Концепция игры для главы 15 от \href{http://lessmilk.com}{Томаса Палефа} (Thomas Palef). Пиксель-арт в главе 16 от Антонио Пердомо Пастора (Antonio Perdomo Pastor).

  Второе издание книги «Выразательный JavaScript» стало возможным благодаря \href{http://eloquentjavascript.net/backers.html}{454 спонсорам}.

  Русский перевод книги выполнил Вячеслав Голованов. Отредактировал русский перевод и скомпилировал эту книгу Дмитрий Лущан.

  \vskip 1em

  \noindent Вы можете купить печатную версию этой книги с дополнительной бонусной главой, напечатанной издательством No Starch Press \url{http://www.amazon.com/gp/product/1593275846/ref=as_li_qf_sp_asin_il_tl?ie=UTF8&camp=1789&creative=9325&creativeASIN=1593275846&linkCode=as2&tag=marijhaver-20&linkId=VPXXXSRYC5COG5R5}.

\tableofcontents

\mainmatter

\input{00_intro.tex}

\input{01_values.tex}

\input{02_program_structure.tex}

\input{03_functions.tex}

\input{04_data.tex}

\input{05_higher_order.tex}

\input{06_object.tex}

\input{07_elife.tex}

\input{08_error.tex}

\input{09_regexp.tex}

\input{10_modules.tex}

\input{11_language.tex}

\input{12_browser.tex}

\input{13_dom.tex}

\input{14_event.tex}

\input{15_game.tex}

\input{16_canvas.tex}

\input{17_http.tex}

\input{18_forms.tex}

\input{19_paint.tex}

\input{20_node.tex}

\input{21_skillsharing.tex}

\input{hints.tex}

\backmatter

\printindex

\end{document}
